
%%%%%%%%%%%%%%%%%%%%%%%%%%%%%%%%%%%%%%%%%%%%%
%%% ! WARNUNG ! %%%%%%%%%%%%%%%%%%%%%%%%%%%%%
%%%%%%%%%%%%%%%%%%%%%%%%%%%%%%%%%%%%%%%%%%%%%
%%%  Diese Datei bitte nur bearbeiten,    %%%
%%%   wenn du ein LaTeX-Experte bist      %%%
%%%             U N D                     %%%
%%%  die Vorlage unbedingt ändern willst  %%%
%%%%%%%%%%%%%%%%%%%%%%%%%%%%%%%%%%%%%%%%%%%%%
%%%%%%%%%%%%%%%%%%%%%%%%%%%%%%%%%%%%%%%%%%%%%
%
%%%%%%%%%%%%%%%%%%%%%%%%%%%%%%%%%%%%%%%%%%%%%%%%%%%%%%%%%%%%%%%%%%%%%%%%%%%%%%%%
%%% DATEI-INFO %%%%%%%%%%%%%%%%%%%%%%%%%%%%%%%%%%%%%%%%%%%%%%%%%%%%%%%%%%%%%%%%%
%%%%%%%%%%%%%%%%%%%%%%%%%%%%%%%%%%%%%%%%%%%%%%%%%%%%%%%%%%%%%%%%%%%%%%%%%%%%%%%%
%%% Diese Datei richtet diverse Pakete und Befehle für die Vorlage ein %%%%%%%%%
%%%%%%%%%%%%%%%%%%%%%%%%%%%%%%%%%%%%%%%%%%%%%%%%%%%%%%%%%%%%%%%%%%%%%%%%%%%%%%%%
%
\def\inPreamble{}% Flag setzen: wird sind in der Präambel
%%%%%%%%%%%%%%%%%%%%%%%%%%%%%%%%%%%%%%%%%%%%%%%%%%%%%%%%%%%%%%%%%%%%%%%%%%%%%%%%
%%% Encoding %%%%%%%%%%%%%%%%%%%%%%%%%%%%%%%%%%%%%%%%%%%%%%%%%%%%%%%%%%%%%%%%%%%

\RequirePackage[utf8]{inputenc}
% UTF8 Codierte .tex-Dateien

\RequirePackage[T1]{fontenc}
% Benutzung der europäischen Schriftkodierung
% z.B. zur richtigen Darstellung von Umlauten


%%%%%%%%%%%%%%%%%%%%%%%%%%%%%%%%%%%%%%%%%%%%%%%%%%%%%%%%%%%%%%%%%%%%%%%%%%%%%%%%
%%% Werkzeuge %%%%%%%%%%%%%%%%%%%%%%%%%%%%%%%%%%%%%%%%%%%%%%%%%%%%%%%%%%%%%%%%%%

\RequirePackage{etoolbox}		% Einfacheres Programmieren in LaTeX
\RequirePackage{calculator}		% Rechnen (auch mit Längen)
\RequirePackage{xfp}			% Genaueres Rechnen (aber nicht mit Längen)
\RequirePackage{pbox}			% \pbox (Parbox flexibler Breite)


%%%%%%%%%%%%%%%%%%%%%%%%%%%%%%%%%%%%%%%%%%%%%%%%%%%%%%%%%%%%%%%%%%%%%%%%%%%%%%%%
%%% Variablen zur Anpassung durch den Benutzer %%%%%%%%%%%%%%%%%%%%%%%%%%%%%%%%%
%%%%%%%%%%%%%%%%%%%%%%%%%%%%%%%%%%%%%%%%%%%%%%%%%%%%%%%%%%%%%%%%%%%%%%%%%%%%%%%%
%%% >>> werden gesetzt in der Datei "PersoenlicheAngaben.tex" %%%%%%%%%%%%%%%%%%

\newbool{abbildungsverzeichnis}
\newbool{quellcodeverzeichnis}
\newbool{tabellenverzeichnis}
\newbool{symbolverzeichnis}
\newbool{akronymverzeichnis}
\newbool{abkuerzungsverzeichnis}
\newbool{glossar}

\newbool{verzeichnisseZusammenfassen}
\newbool{verzeichnisseImInhaltsverzeichnis}

\newbool{nichtZitiertInweiterfuehrendeLiteratur}
\newbool{weiterfuehrendeLiteratur}
\newbool{alleAutorenExplizitNennen}

\newbool{verlaengerung}
\newbool{danksagung}
\newbool{zusatzErklaerung}

\newbool{color}
\newbool{linksMarkieren}

\newbool{doppelseitig}

\newlength{\bindekorrektur}
\setlength{\bindekorrektur}{1cm}

\newlength{\randAussen}
\setlength{\randAussen}{1.88cm} % 1.88 - 2.5cm (einseitig), 2.5 - 3.4 cm (doppelseitig)

\newcommand{\R}{\textcolor{red}}
%%%%%%%%%%%%%%%%%%%%%%%%%%%%%%%%%%%%%%%%%%%%%%%%%%%%%%%%%%%%%%%%%%%%%%%%%%%%%%%%
%%% Benutzer-Einstellungen anwenden %%%%%%%%%%%%%%%%%%%%%%%%%%%%%%%%%%%%%%%%%%%%

%%%%%%%%%%%%%%%%%%%%%%%%%%%%%%%%%%%%%%%%%%%%%%%%%%%%%%%%%%%%%%%%%%%%%%%%%%%%%%%%
%%% DATEI-INFO %%%%%%%%%%%%%%%%%%%%%%%%%%%%%%%%%%%%%%%%%%%%%%%%%%%%%%%%%%%%%%%%%
%%%%%%%%%%%%%%%%%%%%%%%%%%%%%%%%%%%%%%%%%%%%%%%%%%%%%%%%%%%%%%%%%%%%%%%%%%%%%%%%
%%% In dieser Datei werden alle wichtigen Einstellungen der Vorlage gesetzt %%%%
%%% Hier darfst du Werte ändern ;-) %%%%%%%%%%%%%%%%%%%%%%%%%%%%%%%%%%%%%%%%%%%%
%%%%%%%%%%%%%%%%%%%%%%%%%%%%%%%%%%%%%%%%%%%%%%%%%%%%%%%%%%%%%%%%%%%%%%%%%%%%%%%%
%
%%%%%%%%%%%%%%%%%%%%%%%%%%%%%%%%%%%%%%%%%%%%%%%%%%%%%%%%%%%%%%%%%%%%%%%%%%%%%%%%
%%%%%%%%%%%%%%%%%%%%%%%%%%%%%%%%%%%%%%%%%%%%%%%%%%%%%%%%%%%%%%%%%%%%%%%%%%%%%%%%
\ifx\inPreamble\undefined \else %%%% 'MAGIC' %%%%%%%%%%%%%%%%%%%%%%%%%%%%%%%%%%%
%%%%%%%%%%%%%%%%%%%%%%%%%%%%%%%%%%%%%%%%%%%%%%%%%%%%%%%%%%%%%%%%%%%%%%%%%%%%%%%%
%%%%%%%%%%%%%%%%%%%%%%%%%%%%%%%%%%%%%%%%%%%%%%%%%%%%%%%%%%%%%%%%%%%%%%%%%%%%%%%%
%%%%%%% Pflicht-Einstellungen %%%%%%%%%%%%%%%%%%%%%%%%%%%%%%%%%%%%%%%%%%%%%%%%%%

\newcommand{\artderarbeit}{Bachelor-Thesis}
% Bachelor-Thesis, Master-Thesis

\newcommand{\thema}{So lautet das Thema der Thesis}
% Thema der Thesis (wie in der Aufgabenstellung)


\setbool{verlaengerung}{false}
% Gibt es eine Verlängerung der Bearbeitungszeit?
% Wenn ja, hier auf "true" setzen und die Datei Verlaengerung.pdf ersetzen

\setbool{danksagung}{false}
% Eine optionale Danksagung kann in der Datei "Danksagung.tex" formuliert werden


\newcommand{\autor}{Max Mustermann}
% Euer voller Name

\newcommand{\matrikelnummer}{1234567}
% Eure Matrikelnummer

\newcommand{\studiengang}{Informationstechnologie}
% Offizielle Bezeichnung des Studiengangs

\newcommand{\schwerpunkt}{Systems \& Components}
% Wenn es in Eurem Studiengang keine Schwerpunkte gibt einfach leer lassen



\newcommand{\lehrstuhl}{Lehrstuhl für Automatisierungstechnik/Informatik}
% Der Lehrstuhl, an dem die Thesis geschrieben wird

\newcommand{\betreuer}{Vorname Nachname M.Sc.}
% Betreuer? (falls mehrere: mit \\ trennen)

\newcommand{\prueferA}{Prof. Dr.-Ing. Vorname Nachname}
% Erstprüfer (siehe Anmeldung)

\newcommand{\prueferB}{Prof. Dr.-Ing. Vorname Nachname}
% Zweitprüfer (siehe Anmeldung)


\newcommand{\abgabedatum}{03. August 1972}
% Euer Abgabedatum

\newcommand{\ort}{Wuppertal}
% Ort

\newcommand{\schlagwoerter}{Thesis, Bachelor, Bergische Universität Wuppertal}
% Schlagwörter, mit denen man das PDF finden kann


%%% Ende der Pflicht-Einstellungen %%%%%%%%%%%%%%%%%%%%%%%%%%%%%%%%%%%%%%%%%%%%%
%%%%%%%%%%%%%%%%%%%%%%%%%%%%%%%%%%%%%%%%%%%%%%%%%%%%%%%%%%%%%%%%%%%%%%%%%%%%%%%%
%%%%%%%%%%%%%%%%%%%%%%%%%%%%%%%%%%%%%%%%%%%%%%%%%%%%%%%%%%%%%%%%%%%%%%%%%%%%%%%%
%%% Layout-Anpassung (Optional) %%%%%%%%%%%%%%%%%%%%%%%%%%%%%%%%%%%%%%%%%%%%%%%%

\setbool{doppelseitig}{true}			% Einseitiger oder doppelseitiger Druck?

\setbool{linksMarkieren}{true}			% Anklickbare Links im PDF-Dokument markieren?

\setbool{abbildungsverzeichnis}{true} 	% Abbildungsverzeichnis erzeugen?
\setbool{quellcodeverzeichnis}{true}	% Quellcodeverzeichnis erzeugen?
\setbool{tabellenverzeichnis}{true}		% Tabellenverzeichnis erzeugen?
\setbool{symbolverzeichnis}{true}		% Symbolverzeichnis erzeugen?
\setbool{akronymverzeichnis}{true}		% Akronymverzeichnis erzeugen?
\setbool{abkuerzungsverzeichnis}{true}	% Abkürzungsverzeichnis erzeugen?

\setbool{glossar}{true}					% Glossar erzeugen?

\setbool{verzeichnisseImInhaltsverzeichnis}{true} % Verzeichnisse im Inhaltsverzeichnis erwähnen?

\setbool{verzeichnisseZusammenfassen}{true} % Seitenumbruch zwischen den Verzeichnissen deaktivieren?

\setbool{zusatzErklaerung}{false} 		% Hiermit können zusätzliche Erklärungen eingebunden werden (wird im Normalfall nicht benötigt)


%%%%%%%%%%%%%%%%%%%%%%%%%%%%%%%%%%%%%%%%%%%%%%%%%%%%%%%%%%%%%%%%%%%%%%%%%%%%%%%%
%%% Farben (Optional) %%%%%%%%%%%%%%%%%%%%%%%%%%%%%%%%%%%%%%%%%%%%%%%%%%%%%%%%%%

\setbool{color}{true} % Farbe für Design-Elemente verwenden (true/false)

\newcommand{\colormodel}{cmyk} % z.B. cmyk, rgb oder gray
% Falls das Tool eurer Druckerei Schwarz-Weiß-Seiten trotz \setbool{color}{false}
% _fälschlicherweise_ als Farbseiten erkennt (das kann ziemlich teuer werden!), 
% könnt ihr hier das Farbmodell der Vorlage umstellen.
%
% So hatten wir schon einmal den Fall, dass eine Thesis-Druckerei die S/W-Seiten
% erst korrekt erkannt hat, wenn das (eigentlich für das Drucken ungeeignete)
% RGB-Modell verwendet wurde...
% 
% Bilder und eingebundene PDFs werden durch diese Einstellung nicht geändert!
%
% Ein paar Beispiele für sinnvolle Werte:
% {cmyk}  - CMYK (Cyan, Magenta, Yellow, Key) ist _das Farbmodell_ für alles was gedruckt wird.
%           Jede professionelle Druckerei kann damit umgehen!
%           => der Standard bei Drucksachen und daher auch in dieser Vorlage
%
% {rgb}   - RGB (Red, Green, Blue) ist ein Farbmodell für Bildschirme etc.
%           Im Gegensatz zum Druck mit Pigmenten (=subtraktive Farbmischung) wird
%           hier mit Licht gearbeitet (Additive Farbmischung). RGB ist daher nicht
%           für den Druck geeignet und wird vor dem Druckvorgang in ein anderes
%           Farbmodell (z.B. CMYK) umgewandelt.
%           => eigentlich falsch, wird in Einzelfällen aber von Druckereien angefordert
%
% {gray}  - Wer möchte, kann auch das Farbmodell "gray" verwenden. Dann werden alle
%           Farben direkt in Graustufen umgerechnet.
%           In diesem Fall wird zusätzlich \setbool{color}{false} empfohlen.
%           So wird die Darstellung von Quelltexten an die nun fehlenden Farben angepasst.
%
% Weblinks zum Thema:
% https://de.wikipedia.org/wiki/CMYK-Farbmodell
% https://de.wikipedia.org/wiki/RGB-Farbraum
% https://www.ctan.org/pkg/xcolor

%%%%%%%%%%%%%%%%%%%%%%%%%%%%%%%%%%%%%%%%%%%%%%%%%%%%%%%%%%%%%%%%%%%%%%%%%%%%%%%%
%%% Seitenränder %%%%%%%%%%%%%%%%%%%%%%%%%%%%%%%%%%%%%%%%%%%%%%%%%%%%%%%%%%%%%%%

\setlength{\bindekorrektur}{1.0cm}
% Wie breit ist der Teil, an dem die einzelnen Blätter 
% mit einander verklebt/anderweitig verbunden werden?

\setlength{\randAussen}{1.88cm} % 1.88 - 2.5cm (einseitig), 2.5 - 3.4cm (doppelseitig)
% Wie viel Rand außen neben dem Text (einseitig: wie viel Rand rechts neben dem Text)
% Innerer bzw. linker Rand wird daraus automatisch berechnet


%%%%%%%%%%%%%%%%%%%%%%%%%%%%%%%%%%%%%%%%%%%%%%%%%%%%%%%%%%%%%%%%%%%%%%%%%%%%%%%%
%%% Literatur-Anpassung (Optional) %%%%%%%%%%%%%%%%%%%%%%%%%%%%%%%%%%%%%%%%%%%%%
%
% Es gibt drei Arten von Quellen:
%
% Gruppe A : Quellen, die im Text mit \cite{...} zitiert wurden
% Gruppe B : Quellen, die mit \nocite{...} markiert wurden (und sonst nicht zitiert wurden)
% Gruppe C : Quellen, die gar nicht zitiert wurden

\setbool{nichtZitiertInweiterfuehrendeLiteratur}{true}
% nicht zitierte Quellen automatisch mit \nocite{} aufnehmen?
% => d.h. Gruppe C wird automatisch in Gruppe B verschoben 

\setbool{weiterfuehrendeLiteratur}{true}
% Separates Verzeichnis für weiterführende Literatur?
%
% "true": Separates Verzeichnis "Weiterführende Literatur" erstellen
%   Gruppe A => "Literatur"
%   Gruppe B => "Weiterführende Literatur"
% 
% "false": Alles landet in "Literatur"
%   Gruppe A => "Literatur"
%   Gruppe B => "Literatur"

\newcommand{\explizitesNocite}{
	% Wenn bei der Literatur Gruppe B manuell festgelegt werden soll, kann dies hier geschehen:
%	\nocite{ARM:AMBA4AXI4StreamProtocol:v1_0}
%	\nocite{ARM:AMBA_AXI_and_ACE_Protocol_Specification:E}
%	\nocite{AnalogDevices:ADAU1761:rev_C}
%	\nocite{Book:PerceptionBasedDataprocessingInAcoustics}
%	\nocite{Philips:I2S_BUS_Specification}
%	\nocite{Xilinx:PG021:v7_1}
%	\nocite{Xilinx:UG473:v1_11}
%	\nocite{Xilinx:UG761:v13_1}
%	\nocite{Xilinx:UG901:v2016_2}
%	\nocite{Xilinx:UG906:v2016_2}
%	\nocite{Xilinx:WP231}
%	\nocite{Xilinx:XAPP1206:v1_1}
%	\nocite{Xilinx:PG109:v9_0}
%	\nocite{ARM:NEONProgrammersGuide:v1_0}
%	\nocite{Book:TheScientistandEngineersGuidetoDigitalSignalProcessing}
%	\nocite{IEEE:754:2008}
%	\nocite{SpectrumAndSpectralDensityEstimationByTheDFTincludingAListOfWindowFunctions}
}

\setbool{alleAutorenExplizitNennen}{true} % Steuert die Nennung der Autoren im Literaturverzeichnis
% Hier gibt es zwei übliche Varianten:
% true  : Es werden immer alle Autoren genannt.
% false : Bis zu vier Autoren werden mit Namen genannt.
%           Gibt es mehr Autoren, werden die ersten drei genannt und
%           der Rest mit et al. abgekürzt (wie auch im Zitierschlüssel)

%%%%%%%%%%%%%%%%%%%%%%%%%%%%%%%%%%%%%%%%%%%%%%%%%%%%%%%%%%%%%%%%%%%%%%%%%%%%%%%%
%%% Sprach-Anpassung (Optional) %%%%%%%%%%%%%%%%%%%%%%%%%%%%%%%%%%%%%%%%%%%%%%%%
% Eine Liste der mögliche Sprachen finden sich in der Dokumentation zum LaTeX-Paket 'babel'
\newcommand{\hauptsprache}{ngerman} % Sprache, in der das Dokument geschrieben ist. Wird auch für die Verzeichnisnamen etc. verwendet
\newcommand{\weitereSprachen}{ngerman, english, french, latin}% weitere Sprachen, auf die kurzfristig mit \selectlanguage{language} gewechselt wird. (Wenn es mehrere sind werden diese mit Kommata getrennt)

%%%%%%%%%%%%%%%%%%%%%%%%%%%%%%%%%%%%%%%%%%%%%%%%%%%%%%%%%%%%%%%%%%%%%%%%%%%%%%%%
\fi %%%%%%%%%%%%%%%%%%%%%%%%%%%%%%%%%%%%%%%%%%%%%%%%%%%%%%%%%%%%%%%%%%%%%%%%%%%%
%%%%%%%%%%%%%%%%%%%%%%%%%%%%%%%%%%%%%%%%%%%%%%%%%%%%%%%%%%%%%%%%%%%%%%%%%%%%%%%%


%%%%%%%%%%%%%%%%%%%%%%%%%%%%%%%%%%%%%%%%%%%%%%%%%%%%%%%%%%%%%%%%%%%%%%%%%%%%%%%%
%%% Verzeichnis- & PDF-Verzeichnis-Einträge generieren %%%%%%%%%%%%%%%%%%%%%%%%%

%TODO Verwendung so anpassen, dass der Eintrag (=Linkziel) direkt beim entsprechenden Titel steht (=Titel auch mit \verzeichnisEintrag generieren und vom originalverzeichnis ausblenden?)
\newcommand{\verzeichnisEintrag}[2]{%
	% #1 Name
	% #2 Label (für PDF)
	\ifbool{verzeichnisseImInhaltsverzeichnis}{%
%		\ifbool{verzeichnisseZusammenfassen}{%
			%nichts
%		}{%
			\clearpage%
%		}%
		\phantomsection% damit die Seitenzahl auf jeden Fall stimmt
		\nopagebreak%
		\addcontentsline{toc}{chapter}{#1}% 
		\nopagebreak%
	}{%
		\pdfbookmark[0]{#1}{#2}% hier passte die Seitenzahl bisher immer, daher keine \phantomsection nötig
	}%
}


%%%%%%%%%%%%%%%%%%%%%%%%%%%%%%%%%%%%%%%%%%%%%%%%%%%%%%%%%%%%%%%%%%%%%%%%%%%%%%%%
%%% Document-Class %%%%%%%%%%%%%%%%%%%%%%%%%%%%%%%%%%%%%%%%%%%%%%%%%%%%%%%%%%%%%

\ifbool{doppelseitig}{
	\def\onetwoside{twoside}
}{
	\def\onetwoside{oneside}
}

\documentclass[
	\onetwoside,% twoside: zweiseitig, oneside: einseitig
	titlepage, 	% Titelseite steht für sich allein
	12pt,		% Schriftgröße
	a4paper, 	% A4-Papier
]{report}


%%%%%%%%%%%%%%%%%%%%%%%%%%%%%%%%%%%%%%%%%%%%%%%%%%%%%%%%%%%%%%%%%%%%%%%%%%%%%%%%
%%% Sprachen %%%%%%%%%%%%%%%%%%%%%%%%%%%%%%%%%%%%%%%%%%%%%%%%%%%%%%%%%%%%%%%%%%%

\RequirePackage[%
	main=\hauptsprache, %Umsetzung der deutschen Sprache nach neuer Rechtschreibung
	\hauptsprache, % muss hier nochmal erwähnt werden, damit \autoref das auch mitbekommt
	\weitereSprachen %Sprachen, die nur Abschnittsweise (z.B. für Zitate) verwendet werden
	]{babel}

\RequirePackage{translations}

\DeclareTranslation{German}{Bibliography}{Literatur}
\DeclareTranslation{English}{Bibliography}{Bibliography}
\DeclareTranslationFallback{Bibliography}{Bibliography}

\DeclareTranslation{German}{FurtherReading}{Weiterführende Literatur}
\DeclareTranslation{English}{FurtherReading}{Further Reading}
\DeclareTranslationFallback{FurtherReading}{Further Reading}

\DeclareTranslation{German}{abbreviationsname}{Abkürzungen}
\DeclareTranslation{English}{abbreviationsname}{Abbreviations}
\DeclareTranslationFallback{abbreviationsname}{Abbreviations}

%%%%%%%%%%%%%%%%%%%%%%%%%%%%%%%%%%%%%%%%%%%%%%%%%%%%%%%%%%%%%%%%%%%%%%%%%%%%%%%%
%%% Schriftarten, Schriftsatz %%%%%%%%%%%%%%%%%%%%%%%%%%%%%%%%%%%%%%%%%%%%%%%%%%

\RequirePackage{lmodern} % Schriftart lmodern nutzen
\RequirePackage{microtype} % Besserer Schriftsatz

%%%%%%%%%%%%%%%%%%%%%%%%%%%%%%%%%%%%%%%%%%%%%%%%%%%%%%%%%%%%%%%%%%%%%%%%%%%%%%%%
%%% Zitate, Anführungszeichen %%%%%%%%%%%%%%%%%%%%%%%%%%%%%%%%%%%%%%%%%%%%%%%%%%

\RequirePackage[
	autostyle, % Zitierstil an die aktuelle Sprache anpassen
	german=quotes, % wenn Deutsch -> „Anführungszeichen“ benutzen 
]{csquotes} %wird auch von biblatex verwendet

\renewcommand{\mkcitation}[1]{\,#1} % Umdefiniert um doppelklammerung bei \textquote+\cite zu verhindern

%%%%%%%%%%%%%%%%%%%%%%%%%%%%%%%%%%%%%%%%%%%%%%%%%%%%%%%%%%%%%%%%%%%%%%%%%%%%%%%%
%%% Literaturverwaltung %%%%%%%%%%%%%%%%%%%%%%%%%%%%%%%%%%%%%%%%%%%%%%%%%%%%%%%%
\def\autorenAnzahlImLiteraturverzeichnis{4}
\ifbool{alleAutorenExplizitNennen}{
	\def\autorenAnzahlImLiteraturverzeichnis{99}
}{}

\RequirePackage[backend=bibtex8,  %Daten von Bibtex holen
			% style=alphabetic, %In der Form [Xil13] statt [1] referenzieren
			% sorting=nyt,	      %Name,Year,Title 
			% maxbibnames=\autorenAnzahlImLiteraturverzeichnis, %Anzahl maximal/minimal genannter Autoren im Literaturverzeichnis
			% minbibnames=3, 
			% maxcitenames=3, %Anzahl maximal/minimal genannter Autoren mit \citeauthor{...}
			% mincitenames=3,
			% maxsortnames=99, % Anzahl der Autoren, die in die Sortierung eingehen
			% minsortnames=99,
			% maxalphanames=4, % Maximale Anzahl Autoren, bei der jeder Autor mit einem Buchstaben in [ABCD] genannt wird
			% minalphanames=3, % Anzahl Autoren, ab der [ABC+] verwendet wird
			]{biblatex}

\DeclareNameAlias{default}{family-given}% Format: Nachname, Vorname
\DefineBibliographyStrings{ngerman}{ 
	andothers = {{et\,al\adddot}}, % et al.
}
\renewcommand*{\multinamedelim}{\addsemicolon\space} % Zeichen zwischen den Autoren
%\renewcommand*{\finalnamedelim}{\addsemicolon\space} % Zeichen zwischen den letzten beiden Autoren

\addbibresource{Include/Literatur.bib} % Literatur-Datei laden

\DeclareBibliographyCategory{cited} % registrieren, welche Quellen zitiert wurden
\AtEveryCitekey{%
	\addtocategory{cited}{\thefield{entrykey}}%
}

\ifbool{nichtZitiertInweiterfuehrendeLiteratur}{
	\nocite{*} % alle nicht zitierten Quellen mit aufnehmen
}{
	\explizitesNocite
}


%%%%%%%%%%%%%%%%%%%%%%%%%%%%%%%%%%%%%%%%%%%%%%%%%%%%%%%%%%%%%%%%%%%%%%%%%%%%%%%%
%%% Seitengröße festlegen %%%%%%%%%%%%%%%%%%%%%%%%%%%%%%%%%%%%%%%%%%%%%%%%%%%%%%

% Berechnung der seitlichen Seitenränder:
\def\randAussenCm{}
\def\bindekorrekturCm{}
\LENGTHDIVIDE{\randAussen}{1cm}{\randAussenCm} 			% auf cm normieren
\LENGTHDIVIDE{\bindekorrektur}{1cm}{\bindekorrekturCm} 	% auf cm normieren

\newlength{\randRechtsAussen}	% Als Länge anlegen (bessere Wiederverwendbarkeit)
\newlength{\randLinksInnen}

\setlength{\randRechtsAussen}{\fpeval{1*\randAussenCm} cm}
\setlength{\randLinksInnen}{\fpeval{\randAussenCm/2+\bindekorrekturCm} cm}

% Anwenden
\usepackage[
%	showframe, 				% Seitenränder anzeigen - zum Debuggen
	layout=a4paper,			% Papierformat
	headheight=1.2cm,		% durch Kopfzeile festgelegt
	top=3.55cm,				% Abstand Text <-> oberer Seitenrand
	left=\randLinksInnen,	% Abstand Text <-> linker Seitenrand
	right=\randRechtsAussen,% Abstand Text <-> rechter Seitenrand
	bottom=3.7cm 			% Abstand Text <-> unterer Seitenrand
]{geometry}


%%%%%%%%%%%%%%%%%%%%%%%%%%%%%%%%%%%%%%%%%%%%%%%%%%%%%%%%%%%%%%%%%%%%%%%%%%%%%%%%
%%% (R), (C), TM (vor allem für Zitate und das Literaturverzeichnis) %%%%%%%%%%%

\newcommand{\markRegistered}[1]{#1$^\text{\tiny{\textregistered\xspace}}$}
\newcommand{\markCopyrighted}[1]{#1$^\text{\tiny{\copyright\xspace}}$}
\newcommand{\markTrademark}[1]{#1$^\text{\tiny{\texttrademark\xspace}}$}
\usepackage[official]{eurosym}


\usepackage{bbding}%Häkchen für abgehakte Dinge
\newcommand{\ok}{\Checkmark}
\newcommand{\x}{\XSolidBrush}
\newcommand{\xg}{\textcolor{gray}{\x}}


%%%%%%%%%%%%%%%%%%%%%%%%%%%%%%%%%%%%%%%%%%%%%%%%%%%%%%%%%%%%%%%%%%%%%%%%%%%%%%%%
%%% Farben %%%%%%%%%%%%%%%%%%%%%%%%%%%%%%%%%%%%%%%%%%%%%%%%%%%%%%%%%%%%%%%%%%%%%

\usepackage{xcolor}


% Universell nutzbar
\definecolor{Buw}			{cmyk}{0.55, 0.00, 1.00, 0.00} % Pantone 376
\definecolor{BuwEtikett}	{cmyk}{0.55, 0.00, 1.00, 0.10} % Pantone 376 + 10K

\definecolor{BuwFak01}		{cmyk}{0.08, 1.00, 0.65, 0.34} % Pantone 201
\definecolor{BuwFak02}		{cmyk}{1.00, 0.03, 0.34, 0.12} % Pantone 321
\definecolor{BuwFak03}		{cmyk}{1.00, 0.57, 0.12, 0.70} % Pantone 540
\definecolor{BuwFak04_oben}	{cmyk}{0.70, 0.30, 0.00, 0.12} % Pantone 646
\definecolor{BuwFak04_unten}{cmyk}{1.00, 0.61, 0.04, 0.26} % Pantone 301
\definecolor{BuwFak05}		{cmyk}{0.23, 0.17, 0.13, 0.46} % Pantone Cool Gray 8
\definecolor{BuwFak06_oben}	{cmyk}{0.37, 1.00, 0.00, 0.26} % Pantone 2425
\definecolor{BuwFak06_unten}{cmyk}{0.37, 1.00, 0.00, 0.47} % Pantone 2425 + 30K
\definecolor{BuwFak07}		{cmyk}{1.00, 0.00, 0.55, 0.40} % Pantone 562
\definecolor{BuwFak08}		{cmyk}{0.00, 0.00, 0.00, 0.00} %
\definecolor{BuwFak09}		{cmyk}{0.00, 0.30, 0.85, 0.00} % Pantone 1235

\definecolor{pureK}			{cmyk}{0.00, 0.00, 0.00, 1.00} % K

\definecolor{AccentStrong}	{cmyk}{0.000,0.420,0.800,0.100} % Orange
\definecolor{AccentWeak}	{cmyk}{0.295,0.142,0.000,0.310} % Bläuliches Grau

% Farbschema
\ifbool{color}{
	% wenn Farbe
	\colorlet{primaryTop}		{Buw}
	\colorlet{primaryMiddle}	{Buw}
	\colorlet{primaryBottom}	{Buw}
	\colorlet{secondaryTop}		{BuwFak06_oben}
	\colorlet{secondaryMiddle}	{BuwFak06_oben!50!BuwFak06_unten}
	\colorlet{secondaryBottom}	{BuwFak06_unten}
	\colorlet{tertiaryTop}		{black!50}
	\colorlet{tertiaryMiddle}	{black!50}
	\colorlet{tertiaryBottom}	{black!50}
	\colorlet{textBlack}		{black}
}{
	% wenn Graustufen
	\colorlet{primaryTop}		{pureK}
	\colorlet{primaryMiddle}	{pureK}
	\colorlet{primaryBottom}	{pureK}
	\colorlet{secondaryTop}		{pureK!70}
	\colorlet{secondaryMiddle}	{pureK!70}
	\colorlet{secondaryBottom}	{pureK!70}
	\colorlet{tertiaryTop}		{pureK!50}
	\colorlet{tertiaryMiddle}	{pureK!50}
	\colorlet{tertiaryBottom}	{pureK!50}
	\colorlet{textBlack}		{pureK}
}

% Kurzform für mittlere Farbe
\colorlet{primary}			{primaryMiddle}
\colorlet{secondary}		{secondaryMiddle}
\colorlet{tertiary}			{tertiaryMiddle}


% Spezialisierter Anwendungszweck
\colorlet{pagenumber}		{tertiaryBottom!50!textBlack}
\colorlet{headrule}			{primaryTop}
\colorlet{footnoteRule}		{secondary}
\colorlet{footrule}			{primaryBottom}

% TODO Formelnummern Style
\colorlet{captionLabel}		{pureK!80}
\colorlet{subcaptionLabel}	{pureK!70}

\colorlet{captionText}		{pureK!70}
\colorlet{subcaptionText}	{pureK!60}

\colorlet{listingBackground}{pureK!5}

%%%%%%%%%%%%%%%%%%%%%%%%%%%%%%%%%%%%%%%%%%%%%%%%%%%%%%%%%%%%%%%%%%%%%%%%%%%%%%%%
%%% Fußnotengestaltung %%%%%%%%%%%%%%%%%%%%%%%%%%%%%%%%%%%%%%%%%%%%%%%%%%%%%%%%%
%
\renewcommand{\footnoterule}{%
	\color{footnoteRule}{%
%	\textcolor{footnoteRule}{
% 	an der selben Stelle sorgt für Probleme mit umgebrochenen Listings.
%	Das liegt daran, dass \textcolor{}{} zusätzlich zu \color auch noch ein \leavevmode enthält.
%	Das \leavevmode wiederum bringt das Listing durcheinander, so dass z.B. die Zeilennummern direkt beim Umbruch 
%	jeweils vertauscht sind und oberhalb der untersten Zeile der vorderen Seite eine Lücke entsteht.
%	
		\vspace*{-3pt}%
		\hrule height 0.1em width 0.4\columnwidth%
		\vspace*{2.6pt}%
	}%
}%


%%%%%%%%%%%%%%%%%%%%%%%%%%%%%%%%%%%%%%%%%%%%%%%%%%%%%%%%%%%%%%%%%%%%%%%%%%%%%%%%
%%% Kopfzeilen und Fußzeilen, Seitenzahl %%%%%%%%%%%%%%%%%%%%%%%%%%%%%%%%%%%%%%%

\def\customFootrule{
	\renewcommand{\footrulewidth}{1.5mm}
	\renewcommand{\footrule}{%
		\begingroup
			\color{footrule}
			\hrule height \footrulewidth
			\vspace{0.5em}
		\endgroup
	}
}

\def\customHeadrule{
	\renewcommand{\headrulewidth}{1.5mm}
	\renewcommand{\headrule}{%
		\begingroup
			\color{headrule}
			\hrule height \headrulewidth 
		\endgroup
	}
}


\def\renewMarks{
	\renewcommand{\sectionmark}[1]{%
		\markright{%
			##1%~(\thesection)%
		}%
	}
	\renewcommand{\chaptermark}[1]{%
		\markboth{%
			##1% \chaptername~\thechapter%
		}{%
		}%
	}
}

\def\seitenNummer{%
	\color{pagenumber}\sffamily\bfseries\thepage{}%
}

\usepackage{fancyhdr} % Schönere Kopfzeilen
\def\vertikalZentrieren{%
	\vspace{-\footrulewidth}%
	\vspace{-1.275em}%
}

\ifbool{doppelseitig}{
	\def\rightodd{RO}
	\def\righteven{RE}
	\def\leftodd{LO}
	\def\lefteven{LE}
	\def\rightoddlefteven{RO,LE}
	\def\leftoddrighteven{LO,RE}
}{
	\def\rightodd{R}
	\def\righteven{R}
	\def\leftodd{R}
	\def\lefteven{R}
	\def\rightoddlefteven{R}
	\def\leftoddrighteven{L}
}

\fancypagestyle{thesis-page-regular}{%
	\renewMarks
	\fancyhf{}%clear
	\fancyhead[\rightoddlefteven]{%
		\includegraphics[height=2em]{\buwlogopdf}%
	}
	\fancyhead[\leftoddrighteven]{%
		\nouppercase{%
			\sffamily%
			\textbf{\leftmark}%
			\\%
			\rightmark%
		}%
	}
	\fancyfoot[\rightodd]{\vertikalZentrieren\colorbox{white}{\seitenNummer}\hspace{1em}}
	\fancyfoot[\lefteven]{\vertikalZentrieren\hspace{1em}\colorbox{white}{\seitenNummer}}
	
	\customHeadrule
	\customFootrule
}

\fancypagestyle{plain}{%
	\renewMarks
	\fancyhf{} % clear
	\fancyfoot[\rightodd]{\vertikalZentrieren\colorbox{white}{\seitenNummer}\hspace{1em}}
	\fancyfoot[\lefteven]{\vertikalZentrieren\hspace{1em}\colorbox{white}{\seitenNummer}}
	
	\customHeadrule
	\customFootrule
}


%%%%%%%%%%%%%%%%%%%%%%%%%%%%%%%%%%%%%%%%%%%%%%%%%%%%%%%%%%%%%%%%%%%%%%%%%%%%%%%%
%%% Format der Überschriften %%%%%%%%%%%%%%%%%%%%%%%%%%%%%%%%%%%%%%%%%%%%%%%%%%%

\usepackage[%
	nobottomtitles*,	% Überschriften am unteren Seitenrand vermeiden
]{titlesec}
 
\def\titleformatBase{\sffamily\bfseries}
\def\titleformatChapter{\Huge}
\def\titleformatSection{\LARGE}
\def\titleformatSubsection{\Large}
\def\titleformatSubsubsection{\large}
\def\titleformatParagraph{\normalsize}

\def\titlepartChapter	{\titleformatChapter\ifbool{weAreInTheAppendix}{\Alph{chapter}}{\arabic{chapter}}}
\def\titlepartSection	{\titleformatSection\arabic{section}}
\def\titlepartSubsection{\titleformatSubsection\arabic{subsection}}
\newbool{weAreInTheAppendix}
\setbool{weAreInTheAppendix}{false}
\appto{\appendix}{\setbool{weAreInTheAppendix}{true}}
\def\titleNumberExtraDistance{.5cm}

\titleformat{\chapter}[hang]
	{\titleformatBase}
	{\titlepartChapter\hphantom{.\titlepartSection.\titlepartSubsection}}
	{\titleNumberExtraDistance}
	{\titleformatChapter}
	%[after-code]
	
\titleformat{\section}[block]
	{\titleformatBase}
	{\color{textBlack!50}\titlepartChapter.\color{textBlack}\titlepartSection\hphantom{.\titlepartSubsection}}
	{\titleNumberExtraDistance}
	{\titleformatSection}
	%[after-code]

\titleformat{\subsection}[block]
	{\titleformatBase}
	{\color{textBlack!33}\titlepartChapter.\color{textBlack!50}\titlepartSection.\color{textBlack}\titlepartSubsection}
	{\titleNumberExtraDistance}
	{\titleformatSubsection}
	%[after-code]
	
\titleformat{\subsubsection}[block]
	{\titleformatBase}
	{\titleformatSubsubsection\thesubsubsection}
	{}
	{%
%	\hphantom{\titlepartChapter\titlepartSection\titlepartSubsection}\hspace{\titleNumberExtraDistance}%
	\titleformatSubsubsection}
	%[after-code]
	
\titleformat{\paragraph}[runin]
	{\titleformatBase}
	{\titleformatParagraph\theparagraph}
	{0em}
	{%
%	\hphantom{\titlepartChapter\titlepartSection\titlepartSubsection}\hspace{\titleNumberExtraDistance}%
	\titleformatParagraph}%
	[]
	%[after-code]
	
%				command			left	before	after
\titlespacing*{\chapter}		{0pt}	{-2.1em}{1em}
\titlespacing*{\section}		{0pt}	{1em}	{1em}
\titlespacing*{\subsection}		{0em}	{1em}	{1em}
\titlespacing*{\subsubsection}	{0em}	{1em}	{1em}
\titlespacing*{\paragraph}		{0em}	{1em}	{1em}

%%%%%%%%%%%%%%%%%%%%%%%%%%%%%%%%%%%%%%%%%%%%%%%%%%%%%%%%%%%%%%%%%%%%%%%%%%%%%%%%
%%% Grafiken %%%%%%%%%%%%%%%%%%%%%%%%%%%%%%%%%%%%%%%%%%%%%%%%%%%%%%%%%%%%%%%%%%%

\usepackage{graphicx}				% um Bilder einbinden zu können 
\usepackage{pdfpages}				% Ganzseitig externe PDFs einbinden (z.B. für Aufgabenstellung)

\usepackage{tikz}					% Zeichnen per Code


%%%%%%%%%%%%%%%%%%%%%%%%%%%%%%%%%%%%%%%%%%%%%%%%%%%%%%%%%%%%%%%%%%%%%%%%%%%%%%%%
%%% SI-Einheiten und Zahlendarstellung %%%%%%%%%%%%%%%%%%%%%%%%%%%%%%%%%%%%%%%%%

\usepackage[
	% Die Einstellungen für dieses Paket befinden sich in der Datei siunitx.cfg
]{siunitx}


%%%%%%%%%%%%%%%%%%%%%%%%%%%%%%%%%%%%%%%%%%%%%%%%%%%%%%%%%%%%%%%%%%%%%%%%%%%%%%%%
%%% Quellcode %%%%%%%%%%%%%%%%%%%%%%%%%%%%%%%%%%%%%%%%%%%%%%%%%%%%%%%%%%%%%%%%%%
\usepackage{listings} %Für Quelltexte

% manuell Übersetzen
\DeclareTranslation{German}{lstlistingname}{Quellcode}
\DeclareTranslation{English}{lstlistingname}{Source Code}
\DeclareTranslationFallback{lstlistingname}{\lstlistingname}
\renewcommand{\lstlistingname}{\GetTranslation{lstlistingname}}

\DeclareTranslation{German}{lstlistoflistingname}{Quellcodeverzeichnis}
\DeclareTranslation{English}{lstlistoflistingname}{List of Source Codes}
\DeclareTranslationFallback{lstlistoflistingname}{\lstlistlistingname}
\renewcommand{\lstlistlistingname}{\GetTranslation{lstlistoflistingname}}

\ifbool{color}{
	\def\listingBasicstyle		{\color{black}\ttfamily\small}
	\def\listingKeywordstyle	{\color{primary}\bfseries}
	\def\listingStringstyle		{\color{secondary}\slshape}
	\def\listingCommentstyle	{\color{AccentWeak}\fontseries{l}\selectfont}
	\def\listingEmphstyle		{\color{AccentStrong}\bfseries}
	\def\listingRulecolor		{\color{AccentWeak}}
	\def\listingNumbercolor		{\color{AccentWeak!60}}
}{
	\def\listingBasicstyle		{\color{pureK}\ttfamily\small}
	\def\listingKeywordstyle	{\color{pureK}\bfseries}
	\def\listingStringstyle		{\color{pureK}\slshape}
	\def\listingCommentstyle	{\color{pureK}\fontseries{l}\selectfont}
	\def\listingEmphstyle		{\color{pureK}\bfseries\underbar}
	\def\listingRulecolor		{\color{pureK!50}}
	\def\listingNumbercolor		{\color{pureK!26}}
}

\lstset{
	basicstyle	= \listingBasicstyle,		% Basisformat
	keywordstyle= \listingKeywordstyle,		% Schlüsselwörter
	stringstyle	= \listingStringstyle,		% Strings
	commentstyle= \listingCommentstyle,		% Kommentare
	emphstyle	= \listingEmphstyle,		% Hervorhebungen
%	
%	extendedchars=true,      	%Erweiterte Zeichensätze in Listings benutzen (aber nur Single-Byte!)   
	breaklines=true,            % Zeilen werden Umgebrochen
	breakatwhitespace=true,		% erlaube Zeilenumbrüche nur an Whitespace
	breakindent=0em,
	breakautoindent=true, % bei Zeilenumbruch automatisch einrücken
%
	prebreak=,
	postbreak=\mbox{$\hookrightarrow$\hspace{-0.34em}\hphantom{m}},
%	
	showstringspaces=true, 		% Leerzeichen in Strings zeigen
	showspaces=false,           % Leerzeichen anzeigen?
	showtabs=false,             % Tabs anzeigen?
	tabsize=3,					% Zeichenbreite eines Tabs
%	
	keepspaces=false,			% Erlaubt die Unterdrückung von Leerzeichen zugunsten eines besseren Spalten-Layouts
	columns=fixed,				% Zeichenbreite
%	
	captionpos=b, 				% Position der Beschriftung
	xleftmargin=2.2em, 			% Abstand zum Rand
	xrightmargin=0.35em,		%
	framexleftmargin=0em,		% Abstand im Rahmen
	framexrightmargin=0pt,		%
	framexbottommargin=2pt,		%
	framextopmargin=0pt,		%
	frame=tblr,					% Rahmen anzeigen
	frameround=tftf,			% t: abgerundete Ecke, f: eckige Ecke
	backgroundcolor=\color{listingBackground}, % Hintergrundfarbe
	rulecolor=\listingRulecolor,% Farbe der Umrandung
%
	numbers=left,
	numberstyle=\scriptsize\ttfamily\listingNumbercolor, %Stil der Zeilennummern
	stepnumber=1,
}

\lstset{literate= % Deutsche Umlaute und ß in Listings erlauben
	{Ö}{{\"O}}{1}
	{Ä}{{\"A}}{1}
	{Ü}{{\"U}}{1}
	{ß}{{\ss}}{1}
	%{ẞ}{{\MakeUppercase{\ss}}}{1} % KNOWNISSUE: der aktuell verwendete font enthält das zeichen leider nicht, also lieber erstmal weglassen ;-)
	{ü}{{\"u}}{1}
	{ä}{{\"a}}{1}
	{ö}{{\"o}}{1}
}
\linespread{1.25}

\lstdefinelanguage{thesis-latexbeispiel}
	{morekeywords=
		{\\begin, \\end, \\frac, \\pi, \\\\, \\sum, \\varphi, \\pm, \\int, \\limits, \\infty, \\cdot, \\sqrt, \\left, \\right, \\mathrm,
		\\linewidth, \\centering, \\includegraphics, \\caption, \\label, \\lstinline, \\lstinputlisting, \\todo, \\listoftodos, \\url, \\href, \\cite, \\textqoute,
		\\gls, \\glssymbol, \\LaTeX},
		emph={figure, table, tabular, align, lstlisting},
		sensitive=true,
		morecomment=[l]{\%},
%		morecomment=[s]{/*}{*/},
%		morestring=[b]",
		alsoletter={\\},
	}
	

%%%%%%%%%%%%%%%%%%%%%%%%%%%%%%%%%%%%%%%%%%%%%%%%%%%%%%%%%%%%%%%%%%%%%%%%%%%%%%%%
%%% Format der Float-Beschriftungen %%%%%%%%%%%%%%%%%%%%%%%%%%%%%%%%%%%%%%%%%%%%

\usepackage{caption}
\usepackage[labelformat=brace,list=true]{subcaption}
%Farbiger Text geht mit labelfont={bf,it,color=red}
% Mehr Infos zu CaptionStyle http://ctan.math.washington.edu/tex-archive/macros/latex/contrib/caption/caption-eng.pdf
\DeclareCaptionStyle{myCaptionStyle}{%
	format=plain,			% Basis-Format
	labelsep=quad,			% Abstand Typ/Nummer <-> Beschriftung
%	indention=-7em,			%
	margin=0.5em,				% Zusatzabstand Links
	singlelinecheck=false,	%
	labelfont={sf,small,bf,color=captionLabel},% Font Typ/Nummer
	textfont={sf,small, color=captionText},	% Font Beschriftung
%	justification=centering,
}

\DeclareCaptionStyle{mySubCaptionStyle}{%
	format=plain,			% Basis-Format
	labelsep=space,			% Abstand Typ/Nummer <-> Beschriftung
%	indention=-7em,			%
	margin=0.5em,			% Zusatzabstand links
	singlelinecheck=false,	%
	labelfont={sf,small,bf, color=subcaptionLabel},% Font Typ/Nummer
	textfont={sf,small, color=subcaptionText},	% Font Beschriftung
%	justification=centering,
}

% labelformat: empty  :
%              simple : a 
%              brace  : a)
%              parens : (a)


\captionsetup[table]{style=myCaptionStyle}
\captionsetup[figure]{style=myCaptionStyle}
\captionsetup[lstlisting]{style=myCaptionStyle}

\captionsetup[subfigure]{style=mySubCaptionStyle}
\captionsetup[subtable]{style=mySubCaptionStyle}


%%%%%%%%%%%%%%%%%%%%%%%%%%%%%%%%%%%%%%%%%%%%%%%%%%%%%%%%%%%%%%%%%%%%%%%%%%%%%%%%
%%% PDF-Meta-Informationen und Links im Dokument %%%%%%%%%%%%%%%%%%%%%%%%%%%%%%%

\usepackage[]{hyperref}

\ifbool{linksMarkieren}{
	% nichts
}{
	\hypersetup{hidelinks} % Link-Markierung verstecken
}

\hypersetup{
	linktoc=all, 						% Welcher Teil im Inhaltsverzeichnis ist klickbar? (none/section/page/all)
    colorlinks=false,					% false: boxed links (wird nicht mitgedruckt); true: colored links (wird auch so gedruckt!)
	linkcolor=primary!50!white,			% color of internal links
    linkbordercolor=primary!20!white,
%
    citecolor=primary!50!black,			% color of links to bibliography
    citebordercolor=primary!20!gray,
%
    filecolor=AccentStrong,				% color of file links
    filebordercolor=AccentStrong!20,
%
    urlcolor=AccentWeak,				% color of external links
    urlbordercolor=AccentWeak!20,
%
	menucolor=blue,
	menubordercolor=blue,
%
	runcolor=red,
	runbordercolor=red,
%
    %bookmarks=true,			% show bookmarks bar?
    unicode=true,				% non-Latin characters in Acrobat’s bookmarks
    pdftoolbar=true,			% show Acrobat’s toolbar?
    pdfmenubar=true,			% show Acrobat’s menu?
    pdffitwindow=false,			% window fit to page when opened
    pdfstartview={FitH},		% fits the width of the page to the window
    pdftitle=\thema,			% Titel
    pdfauthor=\autor,			% Autor
    pdfsubject=\artderarbeit,	% subject of the document
    pdfcreator=LaTeX,			% creator of the document
    pdfproducer=,				% producer of the document
    pdfkeywords=\schlagwoerter,	% Schlagworte
    pdfnewwindow=true,			% Links in neuem Fenster öffnen
}


%%%%%%%%%%%%%%%%%%%%%%%%%%%%%%%%%%%%%%%%%%%%%%%%%%%%%%%%%%%%%%%%%%%%%%%%%%%%%%%%
%%% Glossar, Abkürzungen, Akronyme, Symbole, ... %%%%%%%%%%%%%%%%%%%%%%%%%%%%%%%

\usepackage[
	toc=false,		% Nicht im Inhaltsverzeichnis anzeigen. Wenn die Verzeichnisse dort erscheinen sollen, machen wir das selber
	translate=babel,% Babel zum Übersetzen verwenden
	acronym,		% Eigenes Verzeichnis für Akronyme	(CRC, BAföG, LASER, ...)
	abbreviations,	% Eigenes Verzeichnis für Abkürzungen (Dr., z.B., bspw., bzw., etc.)
	% symbols,		% Eigenes Verzeichnis für Symbole (µ, ...)
]{glossaries-extra}

\usepackage{glossary-longragged}% Extra Style
\usepackage{glossary-mcols}% Extra Style

% 'seename' wird mit translate=babel nicht übersetzt => dann machen wir das eben selber
\DeclareTranslation{German}{seename}{siehe}
\DeclareTranslation{English}{seename}{see}
\DeclareTranslationFallback{seename}{\seename} 
\renewcommand{\seename}{\expandonce{\GetTranslation{seename}}} %expandonce: es scheint so als würde der Befehl zu früh/spät expandiert - mit expandonce passt es


\makenoidxglossaries %langsames Sortieren, funktioniert aber immer und braucht keine externen Programme
\loadglsentries{Verzeichnisse/Glossar.tex}

%%%%%%%%%%%%%%%%%%%%%%%%%%%%%%%%%%%%%%%%%%%%%%%%%%%%%%%%%%%%%%%%%%%%%%%%%%%%%%%%
%%%%%%%%%%%%%%%%%%%%%%%%%%%%%%%%%%%%%%%%%%%%%%%%%%%%%%%%%%%%%%%%%%%%%%%%%%%%%%%%
% Anpassung von Nummerierungstiefe etc. (default passt aber schon)
%\setcounter{secnumdepth}{3} %Nummerierungstiefe
%\setcounter{tocdepth}{2} %Kapiteltiefe im Inhaltverzeichnis



\usepackage{blindtext}

%%%%%%%%%%%%%%%%%%%%%%%%%%%%%%%%%%%%%%%%%%%%%%%%%%%%%%%%%%%%%%%%%%%%%%%%%%%%%%%%
%%% Todos %%%%%%%%%%%%%%%%%%%%%%%%%%%%%%%%%%%%%%%%%%%%%%%%%%%%%%%%%%%%%%%%%%%%%%
\usepackage[
	ngerman,
%	textwidth=\randRechtsAussen-0em,%TODO sollte sich berechnen lassen, oder?
	textsize=scriptsize
]{todonotes}

\setlength{\marginparwidth}{\randAussen-2.5mm}% hieraus ergibt sich die Breite der Todo-Boxen


%%%%%%%%%%%%%%%%%%%%%%%%%%%%%%%%%%%%%%%%%%%%%%%%%%%%%%%%%%%%%%%%%%%%%%%%%%%%%%%%
%% Für Debugszwecke: Mehr Infos zu Boxen ausgeben:
%\showboxdepth=\maxdimen
%\showboxbreadth=\maxdimen







\usepackage{booktabs}
\usepackage{fancyvrb}
\newcommand{\VerbBar}{|}
\newcommand{\VERB}{\Verb[commandchars=\\\{\}]}
\DefineVerbatimEnvironment{Highlighting}{Verbatim}{commandchars=\\\{\}}



\DefineVerbatimEnvironment{Highlighting}{Verbatim}{commandchars=\\\{\}}
% Add ',fontsize=\small' for more characters per line
\newenvironment{Shaded}{}{}
\newcommand{\AlertTok}[1]{\textcolor[rgb]{1.00,0.00,0.00}{\textbf{#1}}}
\newcommand{\AnnotationTok}[1]{\textcolor[rgb]{0.38,0.63,0.69}{\textbf{\textit{#1}}}}
\newcommand{\AttributeTok}[1]{\textcolor[rgb]{0.49,0.56,0.16}{#1}}
\newcommand{\BaseNTok}[1]{\textcolor[rgb]{0.25,0.63,0.44}{#1}}
\newcommand{\BuiltInTok}[1]{#1}
\newcommand{\CharTok}[1]{\textcolor[rgb]{0.25,0.44,0.63}{#1}}
\newcommand{\CommentTok}[1]{\textcolor[rgb]{0.38,0.63,0.69}{\textit{#1}}}
\newcommand{\CommentVarTok}[1]{\textcolor[rgb]{0.38,0.63,0.69}{\textbf{\textit{#1}}}}
\newcommand{\ConstantTok}[1]{\textcolor[rgb]{0.53,0.00,0.00}{#1}}
\newcommand{\ControlFlowTok}[1]{\textcolor[rgb]{0.00,0.44,0.13}{\textbf{#1}}}
\newcommand{\DataTypeTok}[1]{\textcolor[rgb]{0.56,0.13,0.00}{#1}}
\newcommand{\DecValTok}[1]{\textcolor[rgb]{0.25,0.63,0.44}{#1}}
\newcommand{\DocumentationTok}[1]{\textcolor[rgb]{0.73,0.13,0.13}{\textit{#1}}}
\newcommand{\ErrorTok}[1]{\textcolor[rgb]{1.00,0.00,0.00}{\textbf{#1}}}
\newcommand{\ExtensionTok}[1]{#1}
\newcommand{\FloatTok}[1]{\textcolor[rgb]{0.25,0.63,0.44}{#1}}
\newcommand{\FunctionTok}[1]{\textcolor[rgb]{0.02,0.16,0.49}{#1}}
\newcommand{\ImportTok}[1]{#1}
\newcommand{\InformationTok}[1]{\textcolor[rgb]{0.38,0.63,0.69}{\textbf{\textit{#1}}}}
\newcommand{\KeywordTok}[1]{\textcolor[rgb]{0.00,0.44,0.13}{\textbf{#1}}}
\newcommand{\NormalTok}[1]{#1}
\newcommand{\OperatorTok}[1]{\textcolor[rgb]{0.40,0.40,0.40}{#1}}
\newcommand{\OtherTok}[1]{\textcolor[rgb]{0.00,0.44,0.13}{#1}}
\newcommand{\PreprocessorTok}[1]{\textcolor[rgb]{0.74,0.48,0.00}{#1}}
\newcommand{\RegionMarkerTok}[1]{#1}
\newcommand{\SpecialCharTok}[1]{\textcolor[rgb]{0.25,0.44,0.63}{#1}}
\newcommand{\SpecialStringTok}[1]{\textcolor[rgb]{0.73,0.40,0.53}{#1}}
\newcommand{\StringTok}[1]{\textcolor[rgb]{0.25,0.44,0.63}{#1}}
\newcommand{\VariableTok}[1]{\textcolor[rgb]{0.10,0.09,0.49}{#1}}
\newcommand{\VerbatimStringTok}[1]{\textcolor[rgb]{0.25,0.44,0.63}{#1}}
\newcommand{\WarningTok}[1]{\textcolor[rgb]{0.38,0.63,0.69}{\textbf{\textit{#1}}}}