%%%%%%%%%%%%%%%%%%%%%%%%%%%%%%%%%%%
% Test file for the Wuppertal beamer style
%%%%%%%%%%%%%%%%%%%%%%%%%%%%%%%%%%%
% Karsten Kahl, 15.10.2012
% kkahl@math.uni-wuppertal.de
%%%%%%%%%%%%%%%%%%%%%%%%%%%%%%%%%%%

\documentclass[12pt,rgb]{beamer}
%%% Switch to for top-aligned frames
%\documentclass[12pt,t]{beamer}

\usetheme[vier,talk,biglion,nototal,german]{Wuppertal}
%%%%%%%%%%%%%%%%%%%%%%%%%%%%%%%%%%%
% Options :
%% Style (see outer themes) :
%%% presentation (default): no navigation; outline at begin of each section
%%%           (suppress with section*)
%%% talk : horizontal section navigation in headline
%%% talkext : horizontal section navigation in headline with logo in top left corner
%%% lecture : adds section and subsection navigation in headline
%%%           if #sections or #subsections greater 3 adds logo in top left corner
%% ----------------
%% Color (see color themes) :
%%% vier (default): built around colours of Fakultaet 4
%%% blue : built around bergisch grau-blau
%%%% Additional Color-Options (no longer supported)
%%% highcontrast : high-contrast version of color style (if available)
%% ----------------
%% Logo in lower right corner (see outer themes) :
%%% lion: half lion in color of style
%%% logo (default): BUW logo papperl
%%% nologo: no logo
%% ----------------
%% Miscellaneous:
%%% german: sets main language to german (Part X to Teil X)
%%% nototal: no display of total page numbers ("page/total number of pages" to "page")
%%% imacm: places IMACM papperl on title page and 
%%%        replaces logo in top left corner by IMACM logo (if applicable)
%%%%%%%%%%%%%%%%%%%%%%%%%%%%%%%%%%%

% Algorithm package with beamer friendly option-set
\usepackage[linesnumberedhidden,vlined]{algorithm2e}
% Use Option linesnumbered if you want to have line numbers
\DontPrintSemicolon
\SetKwFor{For}{for}{}{}
\SetKwFor{While}{while}{}{}
% Change to increase or decrease font size or appearance in the algorithm
\SetAlFnt{\footnotesize}

% Used for the TikZ example
\usetikzlibrary{shapes}
% Figure (\ref{TechStackVisualized}) visualizes the tech stack used relative to each other. \R{TODO}

\begin{document}
\title{VSWorkbench: An Extensible Visual Studio Code Plugin for Bridging the Gap between Key Developer Tools}

    \author[Sufyan Dahalan]{Sufyan Dahalan}
    \institute{Bergische Universit\"at Wuppertal}
    \maketitle
    \begin{frame}[t]{Motivation}
        % 2. We need developers with higher efficiency, improving developer's productivity and why 
        \parbox{\textwidth}{
            \itemize{
                \item Context switching detrimental to software developer productivity.
                \item Reducing travel between Visual Studio Code, a universal code editor, and GitLab, a the comprehensive developement operations platform.
                \item VSWorkbench to the rescue
                }
                }
            \end{frame}
            
            \begin{frame}[t]{Einleitung \textrm{II}}
                \parbox{\textwidth}{
                    \itemize{
                        \item Reduce clicks needed to get to 
                    }
% 3. VSWorkbench - Goals and Aims
}
\end{frame}

\begin{frame}[t]{Einleitung \textrm{II}}
    \parbox{\textwidth}{
% 4. VSWorkbench Building Blocks
}
\end{frame}

\begin{frame}[t]{Einleitung \textrm{II}}
    \parbox{\textwidth}{
% 5. VSWorkbench - Capabilities
}
\end{frame}

\begin{frame}[t]{Einleitung \textrm{II}}
    \parbox{\textwidth}{
% 6. VSWorkbench vs GitLab Worflow - Capabilities and Overhead introduced 
}
\end{frame}

\begin{frame}[t]{Einleitung \textrm{II}}
    \parbox{\textwidth}{
% 7. Results, Downloads
}
\end{frame}

\begin{frame}[t]{Einleitung \textrm{II}}
    \parbox{\textwidth}{
% 8. Future goals 
}
\end{frame}

\begin{frame}[t]{Einleitung \textrm{II}}
    \parbox{\textwidth}{
% 9. Thanks for listening
}
\end{frame}

\end{document}
