\chapter{Grundlagen}
	% Hier werden die Grundlagen der Thematik erklärt.
	% Das können z.B. mathematische Grundlagen, Kommunikationsprotokolle oder spezielle Algorithmen sein.
	% Übliches Wissen aus unserer Faktultät wie z.B. die Formel U = R*I kann vorausgesetzt werden.
	%
	% Faustregel: alles, was man selber vorher nicht wusste, aber auch nicht selber entwickelt hat.
	%
	% Hier gilt es aber auch auf Erst- und Zweitgutachter einzugehen.
	% Wenn man weiß, dass einer der beiden ein Thema nicht so genau kennt, sollte es evtl. doch in die Grundlagen.
	%
	% => im Zweifelsfall den Betreuer fragen
	
	
	\section{Verwendete Protokolle}
		\subsection{\texorpdfstring{B$^\text{U}_\text{W}$ 4.0}{BUW 4.0}}
			\blindtext
		\subsection[HTML]{HTML (berühmtes Internetprotokoll)}
			\blindtext
		
		
	\section{Elektrotechnik}
		\subsection{Richtungsabhängigkeit von passiven Bauteilen}
			\blindtext
		\subsection{DaveCAD}
			\blindtext
		
	\section{Mathematik}
		\subsection{Die ganzverwurschtelte Invers-Transformation}
			\blindtext
		\subsection{V\o{}\v{r}w\ae{}r\v{s}\'{e} \c{K}\"{\i}\~{n}\k{e}m\aa{}\c{t}i\c{k}}
			\blindtext
		
	\section{Wirtschaft}
		\subsection{Die Erwerbsregeln}
			\blindtext
			
		\subsection{Toilettenpapier -- das neue Gold?}
			\blindtext

		