%%%%%%%%%%%%%%%%%%%%%%%%%%%%%%%%%%%%%%%%%%%%%%%%%%%%%%%%%%%%%%%%%%%%%%%%%%%%%%%%
%%% DATEI-INFO %%%%%%%%%%%%%%%%%%%%%%%%%%%%%%%%%%%%%%%%%%%%%%%%%%%%%%%%%%%%%%%%%
%%%%%%%%%%%%%%%%%%%%%%%%%%%%%%%%%%%%%%%%%%%%%%%%%%%%%%%%%%%%%%%%%%%%%%%%%%%%%%%%
%%% In dieser Datei werden alle wichtigen Einstellungen der Vorlage gesetzt %%%%
%%% Hier darfst du Werte ändern ;-) %%%%%%%%%%%%%%%%%%%%%%%%%%%%%%%%%%%%%%%%%%%%
%%%%%%%%%%%%%%%%%%%%%%%%%%%%%%%%%%%%%%%%%%%%%%%%%%%%%%%%%%%%%%%%%%%%%%%%%%%%%%%%
%
%%%%%%%%%%%%%%%%%%%%%%%%%%%%%%%%%%%%%%%%%%%%%%%%%%%%%%%%%%%%%%%%%%%%%%%%%%%%%%%%
%%%%%%%%%%%%%%%%%%%%%%%%%%%%%%%%%%%%%%%%%%%%%%%%%%%%%%%%%%%%%%%%%%%%%%%%%%%%%%%%
\ifx\inPreamble\undefined \else %%%% 'MAGIC' %%%%%%%%%%%%%%%%%%%%%%%%%%%%%%%%%%%
%%%%%%%%%%%%%%%%%%%%%%%%%%%%%%%%%%%%%%%%%%%%%%%%%%%%%%%%%%%%%%%%%%%%%%%%%%%%%%%%
%%%%%%%%%%%%%%%%%%%%%%%%%%%%%%%%%%%%%%%%%%%%%%%%%%%%%%%%%%%%%%%%%%%%%%%%%%%%%%%%
%%%%%%% Pflicht-Einstellungen %%%%%%%%%%%%%%%%%%%%%%%%%%%%%%%%%%%%%%%%%%%%%%%%%%

\newcommand{\artderarbeit}{Bachelor-Thesis}
% Bachelor-Thesis, Master-Thesis

\newcommand{\thema}{So lautet das Thema der Thesis}
% Thema der Thesis (wie in der Aufgabenstellung)


\setbool{verlaengerung}{false}
% Gibt es eine Verlängerung der Bearbeitungszeit?
% Wenn ja, hier auf "true" setzen und die Datei Verlaengerung.pdf ersetzen

\setbool{danksagung}{false}
% Eine optionale Danksagung kann in der Datei "Danksagung.tex" formuliert werden


\newcommand{\autor}{Max Mustermann}
% Euer voller Name

\newcommand{\matrikelnummer}{1234567}
% Eure Matrikelnummer

\newcommand{\studiengang}{Informationstechnologie}
% Offizielle Bezeichnung des Studiengangs

\newcommand{\schwerpunkt}{Systems \& Components}
% Wenn es in Eurem Studiengang keine Schwerpunkte gibt einfach leer lassen



\newcommand{\lehrstuhl}{Lehrstuhl für Automatisierungstechnik/Informatik}
% Der Lehrstuhl, an dem die Thesis geschrieben wird

\newcommand{\betreuer}{Vorname Nachname M.Sc.}
% Betreuer? (falls mehrere: mit \\ trennen)

\newcommand{\prueferA}{Prof. Dr.-Ing. Vorname Nachname}
% Erstprüfer (siehe Anmeldung)

\newcommand{\prueferB}{Prof. Dr.-Ing. Vorname Nachname}
% Zweitprüfer (siehe Anmeldung)


\newcommand{\abgabedatum}{03. August 1972}
% Euer Abgabedatum

\newcommand{\ort}{Wuppertal}
% Ort

\newcommand{\schlagwoerter}{Thesis, Bachelor, Bergische Universität Wuppertal}
% Schlagwörter, mit denen man das PDF finden kann


%%% Ende der Pflicht-Einstellungen %%%%%%%%%%%%%%%%%%%%%%%%%%%%%%%%%%%%%%%%%%%%%
%%%%%%%%%%%%%%%%%%%%%%%%%%%%%%%%%%%%%%%%%%%%%%%%%%%%%%%%%%%%%%%%%%%%%%%%%%%%%%%%
%%%%%%%%%%%%%%%%%%%%%%%%%%%%%%%%%%%%%%%%%%%%%%%%%%%%%%%%%%%%%%%%%%%%%%%%%%%%%%%%
%%% Layout-Anpassung (Optional) %%%%%%%%%%%%%%%%%%%%%%%%%%%%%%%%%%%%%%%%%%%%%%%%

\setbool{doppelseitig}{true}			% Einseitiger oder doppelseitiger Druck?

\setbool{linksMarkieren}{true}			% Anklickbare Links im PDF-Dokument markieren?

\setbool{abbildungsverzeichnis}{true} 	% Abbildungsverzeichnis erzeugen?
\setbool{quellcodeverzeichnis}{true}	% Quellcodeverzeichnis erzeugen?
\setbool{tabellenverzeichnis}{true}		% Tabellenverzeichnis erzeugen?
\setbool{symbolverzeichnis}{true}		% Symbolverzeichnis erzeugen?
\setbool{akronymverzeichnis}{true}		% Akronymverzeichnis erzeugen?
\setbool{abkuerzungsverzeichnis}{true}	% Abkürzungsverzeichnis erzeugen?

\setbool{glossar}{true}					% Glossar erzeugen?

\setbool{verzeichnisseImInhaltsverzeichnis}{true} % Verzeichnisse im Inhaltsverzeichnis erwähnen?

\setbool{verzeichnisseZusammenfassen}{true} % Seitenumbruch zwischen den Verzeichnissen deaktivieren?

\setbool{zusatzErklaerung}{false} 		% Hiermit können zusätzliche Erklärungen eingebunden werden (wird im Normalfall nicht benötigt)


%%%%%%%%%%%%%%%%%%%%%%%%%%%%%%%%%%%%%%%%%%%%%%%%%%%%%%%%%%%%%%%%%%%%%%%%%%%%%%%%
%%% Farben (Optional) %%%%%%%%%%%%%%%%%%%%%%%%%%%%%%%%%%%%%%%%%%%%%%%%%%%%%%%%%%

\setbool{color}{true} % Farbe für Design-Elemente verwenden (true/false)

\newcommand{\colormodel}{cmyk} % z.B. cmyk, rgb oder gray
% Falls das Tool eurer Druckerei Schwarz-Weiß-Seiten trotz \setbool{color}{false}
% _fälschlicherweise_ als Farbseiten erkennt (das kann ziemlich teuer werden!), 
% könnt ihr hier das Farbmodell der Vorlage umstellen.
%
% So hatten wir schon einmal den Fall, dass eine Thesis-Druckerei die S/W-Seiten
% erst korrekt erkannt hat, wenn das (eigentlich für das Drucken ungeeignete)
% RGB-Modell verwendet wurde...
% 
% Bilder und eingebundene PDFs werden durch diese Einstellung nicht geändert!
%
% Ein paar Beispiele für sinnvolle Werte:
% {cmyk}  - CMYK (Cyan, Magenta, Yellow, Key) ist _das Farbmodell_ für alles was gedruckt wird.
%           Jede professionelle Druckerei kann damit umgehen!
%           => der Standard bei Drucksachen und daher auch in dieser Vorlage
%
% {rgb}   - RGB (Red, Green, Blue) ist ein Farbmodell für Bildschirme etc.
%           Im Gegensatz zum Druck mit Pigmenten (=subtraktive Farbmischung) wird
%           hier mit Licht gearbeitet (Additive Farbmischung). RGB ist daher nicht
%           für den Druck geeignet und wird vor dem Druckvorgang in ein anderes
%           Farbmodell (z.B. CMYK) umgewandelt.
%           => eigentlich falsch, wird in Einzelfällen aber von Druckereien angefordert
%
% {gray}  - Wer möchte, kann auch das Farbmodell "gray" verwenden. Dann werden alle
%           Farben direkt in Graustufen umgerechnet.
%           In diesem Fall wird zusätzlich \setbool{color}{false} empfohlen.
%           So wird die Darstellung von Quelltexten an die nun fehlenden Farben angepasst.
%
% Weblinks zum Thema:
% https://de.wikipedia.org/wiki/CMYK-Farbmodell
% https://de.wikipedia.org/wiki/RGB-Farbraum
% https://www.ctan.org/pkg/xcolor

%%%%%%%%%%%%%%%%%%%%%%%%%%%%%%%%%%%%%%%%%%%%%%%%%%%%%%%%%%%%%%%%%%%%%%%%%%%%%%%%
%%% Seitenränder %%%%%%%%%%%%%%%%%%%%%%%%%%%%%%%%%%%%%%%%%%%%%%%%%%%%%%%%%%%%%%%

\setlength{\bindekorrektur}{1.0cm}
% Wie breit ist der Teil, an dem die einzelnen Blätter 
% mit einander verklebt/anderweitig verbunden werden?

\setlength{\randAussen}{1.88cm} % 1.88 - 2.5cm (einseitig), 2.5 - 3.4cm (doppelseitig)
% Wie viel Rand außen neben dem Text (einseitig: wie viel Rand rechts neben dem Text)
% Innerer bzw. linker Rand wird daraus automatisch berechnet


%%%%%%%%%%%%%%%%%%%%%%%%%%%%%%%%%%%%%%%%%%%%%%%%%%%%%%%%%%%%%%%%%%%%%%%%%%%%%%%%
%%% Literatur-Anpassung (Optional) %%%%%%%%%%%%%%%%%%%%%%%%%%%%%%%%%%%%%%%%%%%%%
%
% Es gibt drei Arten von Quellen:
%
% Gruppe A : Quellen, die im Text mit \cite{...} zitiert wurden
% Gruppe B : Quellen, die mit \nocite{...} markiert wurden (und sonst nicht zitiert wurden)
% Gruppe C : Quellen, die gar nicht zitiert wurden

\setbool{nichtZitiertInweiterfuehrendeLiteratur}{true}
% nicht zitierte Quellen automatisch mit \nocite{} aufnehmen?
% => d.h. Gruppe C wird automatisch in Gruppe B verschoben 

\setbool{weiterfuehrendeLiteratur}{true}
% Separates Verzeichnis für weiterführende Literatur?
%
% "true": Separates Verzeichnis "Weiterführende Literatur" erstellen
%   Gruppe A => "Literatur"
%   Gruppe B => "Weiterführende Literatur"
% 
% "false": Alles landet in "Literatur"
%   Gruppe A => "Literatur"
%   Gruppe B => "Literatur"

\newcommand{\explizitesNocite}{
	% Wenn bei der Literatur Gruppe B manuell festgelegt werden soll, kann dies hier geschehen:
%	\nocite{ARM:AMBA4AXI4StreamProtocol:v1_0}
%	\nocite{ARM:AMBA_AXI_and_ACE_Protocol_Specification:E}
%	\nocite{AnalogDevices:ADAU1761:rev_C}
%	\nocite{Book:PerceptionBasedDataprocessingInAcoustics}
%	\nocite{Philips:I2S_BUS_Specification}
%	\nocite{Xilinx:PG021:v7_1}
%	\nocite{Xilinx:UG473:v1_11}
%	\nocite{Xilinx:UG761:v13_1}
%	\nocite{Xilinx:UG901:v2016_2}
%	\nocite{Xilinx:UG906:v2016_2}
%	\nocite{Xilinx:WP231}
%	\nocite{Xilinx:XAPP1206:v1_1}
%	\nocite{Xilinx:PG109:v9_0}
%	\nocite{ARM:NEONProgrammersGuide:v1_0}
%	\nocite{Book:TheScientistandEngineersGuidetoDigitalSignalProcessing}
%	\nocite{IEEE:754:2008}
%	\nocite{SpectrumAndSpectralDensityEstimationByTheDFTincludingAListOfWindowFunctions}
}

\setbool{alleAutorenExplizitNennen}{true} % Steuert die Nennung der Autoren im Literaturverzeichnis
% Hier gibt es zwei übliche Varianten:
% true  : Es werden immer alle Autoren genannt.
% false : Bis zu vier Autoren werden mit Namen genannt.
%           Gibt es mehr Autoren, werden die ersten drei genannt und
%           der Rest mit et al. abgekürzt (wie auch im Zitierschlüssel)

%%%%%%%%%%%%%%%%%%%%%%%%%%%%%%%%%%%%%%%%%%%%%%%%%%%%%%%%%%%%%%%%%%%%%%%%%%%%%%%%
%%% Sprach-Anpassung (Optional) %%%%%%%%%%%%%%%%%%%%%%%%%%%%%%%%%%%%%%%%%%%%%%%%
% Eine Liste der mögliche Sprachen finden sich in der Dokumentation zum LaTeX-Paket 'babel'
\newcommand{\hauptsprache}{ngerman} % Sprache, in der das Dokument geschrieben ist. Wird auch für die Verzeichnisnamen etc. verwendet
\newcommand{\weitereSprachen}{ngerman, english, french, latin}% weitere Sprachen, auf die kurzfristig mit \selectlanguage{language} gewechselt wird. (Wenn es mehrere sind werden diese mit Kommata getrennt)

%%%%%%%%%%%%%%%%%%%%%%%%%%%%%%%%%%%%%%%%%%%%%%%%%%%%%%%%%%%%%%%%%%%%%%%%%%%%%%%%
\fi %%%%%%%%%%%%%%%%%%%%%%%%%%%%%%%%%%%%%%%%%%%%%%%%%%%%%%%%%%%%%%%%%%%%%%%%%%%%
%%%%%%%%%%%%%%%%%%%%%%%%%%%%%%%%%%%%%%%%%%%%%%%%%%%%%%%%%%%%%%%%%%%%%%%%%%%%%%%%